%
% @author   Shmish  "shmish90@gmail.com"
% @legal    MIT     "(c) Christopher Schmitt"
%


\documentclass{article}


%
% Document Imports
%

\usepackage{fancyhdr}
\usepackage{extramarks}
\usepackage{amsmath}
\usepackage{amssymb}
\usepackage{amsthm}
\usepackage{amsfonts}
\usepackage{color}
\usepackage{tikz}



%
% Document Configuation
%

\newcommand{\hwAuthor}{Christopher Schmitt}
\newcommand{\hwSubject}{Math 218}
\newcommand{\hwSection}{Section 81}
\newcommand{\hwSemester}{Summer 2019}
\newcommand{\hwAssignment}{Assignment 6}


%
% Document Enviornments
%

\setlength{\headheight}{65pt}
\pagestyle{fancy}
\lhead{\hwAuthor}
\rhead{
  \hwSubject \\
  \hwSection \\
  \hwSemester \\
  \hwAssignment
}

\newenvironment{problem}[1]{
  \nobreak\section*{Problem #1}
}{}


%
% Document Start
%

\begin{document}
  \begin{problem}{1}
    Use the binary exponentiation algorithm to compute the remainder when $3^{85}$ is divided by $53$
    \begin{equation*}
      \begin{split}
        3^{2} & \equiv 9 (mod 53)\\
        3^{4} & \equiv 28 (mod 53)\\
        & .\\
        & .\\
        & 7
      \end{split}
    \end{equation*}
  \end{problem}

  \begin{problem}{2}
    \textbf{(a)} Find the inverse (reciprocal) of $21$, modulo $919$.
    \begin{center}
      $4 \times 919 + 21(744) \equiv 1 (mod 919)$\\
      $744$
    \end{center}
    \textbf{(b)} Use your answer to part (a) to solve the congruence equation.
    \begin{center}
      $21x \equiv 13 (mod 919)$\\
      $21(744) \equiv 1 (mod 919)$\\
      $21(9674) \equiv 13 (mod 919)$\\
      $21(484) \equiv 13 (mod 919)$
    \end{center}
  \end{problem}

  \begin{problem}{3}
    Use the Principle of Mathematical Induction to prove the following formula:
    \begin{proof}
      Basis Step: $n = 1$\\
      LHS = $1 \times 10 = 10$\\
      RHS = $\frac{1(1 + 1)(1 + 14)}{3} = 10$\\
      Inductive Step: $n = k$\\
      $1(10) + 2(11) + 3(12) + \dots + k(k + 9)$\\
      Consider $n = k + 1$\\
      LHS = $1(10) + 2(11) + 3(12) + \dots + k(k + 9) + (k + 1)(k + 10)$\\
      LHS = $\frac{k(k + 1)(k + 14)}{3} + (k + 1)(k + 10)$\\
      LHS = $\frac{k(k + 1)(k + 14) + 3(k + 1)(k + 10)}{3}$\\
      LHS = $\frac{k^3 + 18k^2 + 47k + 30}{3}$\\
      RHS = $\frac{(k+1)(k+2)(k+15)}{3}$\\
      RHS = $\frac{k^3 + 18k^2 + 47k + 30}{3}$
    \end{proof}
  \end{problem}

  \begin{problem}{4}
    Use the Principle of Mathematical Induction to prove that:
    \begin{center}
      $7$ is a divisor of $6 * 4^n + 11^n$, for all integers $n \ge 0$.
    \end{center}
    \begin{proof}
      Basis Step: n = 1\\
      $7(u) = 6 * 4^n + 11^n$, where $u \in \mathbf{Z}$\\
      $7(u) = 6 * 4 + 11 = 35 = 7(5)$\\
      Inductive Step: n = k\\
      $7(u) = 6 * 4^k + 11^k$\\
      $7(u) = 6 * 44^k$\\
      Consider n = k + 1\\
      $7(u) = 6 * 4^{k + 1} + 11^{k + 1}$\\
      $7(u) = 6 * 44^{k + 1}$
    \end{proof}
  \end{problem}
\end{document}
