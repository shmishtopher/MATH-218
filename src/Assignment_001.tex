%
% @author   Shmish  "shmish90@gmail.com"
% @legal    MIT     "(c) Christopher Schmitt"
%


\documentclass{article}


%
% Document Imports
%

\usepackage{fancyhdr}
\usepackage{extramarks}
\usepackage{amsmath}
\usepackage{amssymb}
\usepackage{amsthm}
\usepackage{amsfonts}



%
% Document Configuation
%

\newcommand{\hwAuthor}{Christopher Schmitt}
\newcommand{\hwSubject}{Math 218}
\newcommand{\hwSection}{Section 81}
\newcommand{\hwSemester}{Summer 2019}
\newcommand{\hwAssignment}{Assignment 1}


%
% Document Enviornments
%

\setlength{\headheight}{65pt}
\pagestyle{fancy}
\lhead{\hwAuthor}
\rhead{
  \hwSubject \\
  \hwSection \\
  \hwSemester \\
  \hwAssignment
}

\newenvironment{problem}[1]{
  \nobreak\section*{Problem #1}
}{}


%
% Document Start
%

\begin{document}
  \begin{problem}{1}
    Are the following statements TRUE or FALSE? Explain why.
    
    \begin{enumerate}
      \item[(a)] If $\pi$ is rational, then so is $2$
      \item[(b)] If $\pi$ is irrational, then so is $2$
    \end{enumerate}
    
    \noindent\textbf{Solution}

    \begin{enumerate}
      \item[(a)] True, since $False \rightarrow True$
      \item[(b)] False, since $True \nrightarrow False$ 
    \end{enumerate}
  \end{problem}

  \begin{problem}{2}
    Consider the statement: “If an animal is an rhinoceros, then it has a horn.”

    \begin{enumerate}
      \item[(a)] Write down the CONVERSE of this statement.
      \item[(b)] Write down the CONTRAPOSITIVE of this statement. 
    \end{enumerate}

    \noindent\textbf{Solution}

    \begin{enumerate}
      \item[(a)] "If an animal has a horn, then it is a rhinoceros"
      \item[(b)] "If an animal does not have a horn, then it is not a rhinoceros" 
    \end{enumerate}
  \end{problem}

  \begin{problem}{3}
    Let $x$ be a real number. Using the definition of rational number, write a direct proof of the following: If $x$ is rational, then $x^2 + 5$ is also rational.

    \begin{proof}
      Let $x$ be a rational number.
      \begin{equation*}
        \begin{split}
          x & = \frac{a}{b}\text{, where $a, b$ are integers and $b \neq 0$ }\\
          x^2 & = \frac{a^2}{b^2}\\
          x^2 + 5 & = \frac{a^2}{b^2} + 5\\
          x^2 + 5 & = \frac{a^2 + 5b^2}{b^2}\text{, which is rational.}
        \end{split}
      \end{equation*}
    \end{proof}
  \end{problem}

  \begin{problem}{4}
    Let $x$ be a positive real number. Using the definition of a rational number, write a proof by contraposition of the following: If $x$ irrational, then $\sqrt{x + 6}$ is also irrational.
    
    \begin{proof}
      (By contraposition) Let $\sqrt{x + 6}$ be rational.
      \begin{equation*}
        \begin{split}
          \sqrt{x + 6} & = \frac{a}{b}\text{, where $a, b$ are integers and $b \neq 0$}\\
          x + 6 & = \frac{a^2}{b^2}\\
          x & = \frac{a^2}{b^2} - 6\\
          x & = \frac{a^2 - 6b^2}{b^2}\text{, which is rational}
        \end{split}
      \end{equation*}
    \end{proof}
  \end{problem}

  \begin{problem}{5}
    let $n$ be an integer. Using the definition of odd/even, write a proof of the following: $n$ is even if and only if $2n^2 + 5n + 7$ is odd.

    \begin{proof}($p \Rightarrow q$) Suppose that $n$ is even.
      \begin{equation*}
        \begin{split}
          n & = 2(k)\text{, where $k$ is an integer}\\
          2n^2 + 5n + 7 & = 2(2k)^2 + 5(2k) + 7\\
          2n^2 + 5n + 7 & = 8k^2 + 10k + 7\\
          2n^2 + 5n + 7 & = 8k^2 + 10k + 6 + 1\\
          2n^2 + 5n + 7 & = 2(4k^2 + 5k + 3) + 1\text{, which is odd.}
        \end{split}
      \end{equation*}
      
      ($\neg p \Rightarrow \neg q$) Suppose that $n$ is odd.
      \begin{equation*}
        \begin{split}
          n & = 2(k) + 1\text{, where $k$ is an integer}\\
          2n^2 + 5n + 7 & = 2(2k + 1)^2 + 5(2k + 1) + 7\\
          2n^2 + 5n + 7 & = 8k^2 + 18k + 14\\
          2n^2 + 5n + 7 & = 2(4k^2 + 9k + 7)\text{, which is even.}
        \end{split}
      \end{equation*}
    \end{proof}
  \end{problem}

  \begin{problem}{6}
    Using the definition of odd and even, write a proof of the following: $n^2 + 3n + 7$ is odd.

    \begin{proof}(By cases) Suppose that $n$ is even.
      \begin{equation*}
        \begin{split}
          n & = 2(k)\text{, where k is an integer}\\
          n^2 + 3n + 7 & = (2k)^2 + 3(2k) + 7\\
          n^2 + 3n + 7 & = 4k^2 + 6k + 7\\
          n^2 + 3n + 7 & = 4k^2 + 6k + 6 + 1\\
          n^2 + 3n + 7 & = 2(2k^2 + 3k + 3) + 1\text{, which is odd.}
        \end{split}
      \end{equation*}
      
      Suppose that $n$ is odd.
      \begin{equation*}
        \begin{split}
          n & = 2(k) + 1\text{, where k is an integer}\\
          n^2 + 3n + 7 & = (2k + 1)^2 + 3(2k + 1) + 7\\
          n^2 + 3n + 7 & = 4k^2 + 10k + 11\\
          n^2 + 3n + 7 & = 4k^2 + 10k + 10 + 1\\
          n^2 + 3n + 7 & = 2(2k^2 + 5k + 5) + 1\text{, which is odd.}\\
        \end{split}
      \end{equation*}
    \end{proof}
  \end{problem}

  \begin{problem}{7}
    Prove that there exists positive integers $a, b$ such that $a^2 + b^2 = 100$

    \begin{proof}(By example)
      \begin{equation*}
        6^2 + 8^2 = 100
      \end{equation*}
    \end{proof}
  \end{problem}
\end{document}
