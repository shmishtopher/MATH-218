%
% @author   Shmish  "shmish90@gmail.com"
% @legal    MIT     "(c) Christopher Schmitt"
%


\documentclass{article}


%
% Document Imports
%

\usepackage{fancyhdr}
\usepackage{extramarks}
\usepackage{amsmath}
\usepackage{amssymb}
\usepackage{amsthm}
\usepackage{amsfonts}
\usepackage{color}



%
% Document Configuation
%

\newcommand{\hwAuthor}{Christopher Schmitt}
\newcommand{\hwSubject}{Math 218}
\newcommand{\hwSection}{Section 81}
\newcommand{\hwSemester}{Summer 2019}
\newcommand{\hwAssignment}{Assignment 2}


%
% Document Enviornments
%

\setlength{\headheight}{65pt}
\pagestyle{fancy}
\lhead{\hwAuthor}
\rhead{
  \hwSubject \\
  \hwSection \\
  \hwSemester \\
  \hwAssignment
}

\newenvironment{problem}[1]{
  \nobreak\section*{Problem #1}
}{}


%
% Document Start
%

\begin{document}
  \begin{problem}{1}
    Let $x$ be a fixed positive real number. Prove that there exists a unique positive real number $y$ such that $xy = 3$.
    \begin{proof} Let $x$ be a fixed positive real.
      \begin{equation*}
        \begin{split}
          xy & = 3\\
          \frac{xy}{x} & = \frac{3}{x}\\
          y & = \frac{3}{x}\text{, which is a positive real.}
        \end{split}
      \end{equation*}
      Suppose $x(y_1) = 3$\\
      Suppose $x(y_2) = 3$
      \begin{equation*}
        \begin{split}
          x(y_1) & = x(y_2)\\
          \frac{x(y_1)}{x} & = \frac{x(y_2)}{x}\\
          y_1 & = y_2\text{, thus any two solution are the same.}
        \end{split}
      \end{equation*}
    \end{proof}
  \end{problem}

  \begin{problem}{2}
    Write the negation of the following statement in English. Do not simply add the words “not” or “it is not the case that” or similar before the existing statement.
    \begin{center}
      “Every dog likes some flavor of Brand XYZ dog food.”
    \end{center}
    \textbf{Solution}
    \begin{center}
      “Some dog does not like any flavor of Brand XYZ dog food.”
    \end{center}
  \end{problem}

  \begin{problem}{3}
    Consider the following compound statements:
    \begin{center}
      $(p \implies q) \wedge (\neg r \implies q)$ and $(p \vee \neg r) \implies q$
    \end{center}
    are logically equivalent by constructing truth tables.
    \begin{center}
    \begin{tabular}{@{ }c@{ }@{ }c@{ }@{ }c | c@{ }@{}c@{}@{ }c@{ }@{ }c@{ }@{ }c@{ }@{}c@{}@{ }c@{ }@{}c@{}@{ }c@{ }@{ }c@{ }@{ }c@{ }@{ }c@{ }@{}c@{}@{ }c}
      p & q & r &  & ( & p & $\rightarrow$ & q & ) & $\&$ & ( & $\sim$ & r & $\rightarrow$ & q & ) & \\
      \hline 
      T & T & T &  &  & T & T & T &  & \textcolor{red}{T} &  & F & T & T & T &  & \\
      T & T & F &  &  & T & T & T &  & \textcolor{red}{T} &  & T & F & T & T &  & \\
      T & F & T &  &  & T & F & F &  & \textcolor{red}{F} &  & F & T & T & F &  & \\
      T & F & F &  &  & T & F & F &  & \textcolor{red}{F} &  & T & F & F & F &  & \\
      F & T & T &  &  & F & T & T &  & \textcolor{red}{T} &  & F & T & T & T &  & \\
      F & T & F &  &  & F & T & T &  & \textcolor{red}{T} &  & T & F & T & T &  & \\
      F & F & T &  &  & F & T & F &  & \textcolor{red}{T} &  & F & T & T & F &  & \\
      F & F & F &  &  & F & T & F &  & \textcolor{red}{F} &  & T & F & F & F &  & \\
    \end{tabular}
    \end{center}
    \begin{center}
      \begin{tabular}{@{ }c@{ }@{ }c@{ }@{ }c | c@{ }@{}c@{}@{ }c@{ }@{ }c@{ }@{ }c@{ }@{ }c@{ }@{}c@{}@{ }c@{ }@{ }c@{ }@{ }c}
        p & q & r &  & ( & p & $\lor$ & $\sim$ & r & ) & $\rightarrow$ & q & \\
        \hline 
        T & T & T &  &  & T & T & F & T &  & \textcolor{red}{T} & T & \\
        T & T & F &  &  & T & T & T & F &  & \textcolor{red}{T} & T & \\
        T & F & T &  &  & T & T & F & T &  & \textcolor{red}{F} & F & \\
        T & F & F &  &  & T & T & T & F &  & \textcolor{red}{F} & F & \\
        F & T & T &  &  & F & F & F & T &  & \textcolor{red}{T} & T & \\
        F & T & F &  &  & F & T & T & F &  & \textcolor{red}{T} & T & \\
        F & F & T &  &  & F & F & F & T &  & \textcolor{red}{T} & F & \\
        F & F & F &  &  & F & T & T & F &  & \textcolor{red}{F} & F & \\
        \end{tabular}
    \end{center}
  \end{problem}

  \begin{problem}{4}
    Prove that the compound statements in Problem 3 are logically equivalent by using the basic logical equivalences. At each step, state which basic logical equivalence you are using.
    \begin{enumerate}
      \item $(p \implies q) \wedge (\neg r \implies q)$
    \end{enumerate}
    Since $(p \implies q) \wedge (r \implies q) \equiv (p \vee r) \implies q$, we can skip right to:
    \begin{enumerate}
      \item[2.] $(p \vee \neg r) \implies q$ 
    \end{enumerate}
  \end{problem}

  \begin{problem}{5}
    Write the compound statements in Problem 3 in disjunctive normal form. (Since they are logically equivalent, they have the same disjunctive normal form, so you only need to give one answer.
    \begin{center}
      $(p \wedge q \wedge r) \vee (p \wedge q \wedge \neg r) \vee (\neg p \wedge q \wedge r) \vee (\neg p \wedge q \wedge \neg r) \vee (\neg p \wedge \neg q \wedge r)$
    \end{center}
  \end{problem}

  \begin{problem}{6}
    Consider the premises:
    \begin{enumerate}
      \item It is not snowing today and it is windy;
      \item School will be canceled only if it is snowing today;
      \item If school is not canceled today, then our study group will meet;
      \item If I did not get A on the exam, then our study group did not meet.
    \end{enumerate}
    Do these premises imply the conclusion “I will get A or B on the exam”? If so, explain why by using the rules of inference. If not, explain why not.
    \begin{enumerate}
      \item $\neg S \wedge W$
      \item $C \iff S$
      \item $\neg C \implies G$
      \item $\neg A \implies \neg G$
    \end{enumerate}
    \textbf{Steps}
    \begin{enumerate}
      \item Contrapositive of 4: $G \implies A$
      \item School is not canceled because $C \iff S$ and $\neg S \wedge W$
      \item School is not canceled, so the group meets because $\neg C \implies G$
      \item And $G \implies A$ as seen earlier. So $A \vee B$ is true because $G$ is true.
    \end{enumerate}
  \end{problem}
\end{document}
